% Contributions are much appreciated, in order to contribute to this project, head over to this repository:
% https://github.com/bshramin/uofa-eng-assignment

\documentclass[11pt,letterpaper]{article}
\textwidth 6.5in
\textheight 9.in
\oddsidemargin 0in
\headheight 0in
\usepackage{graphicx}
\usepackage{fancybox}
\usepackage[utf8]{inputenc}
\usepackage{epsfig,graphicx}
\usepackage{multicol,pst-plot}
\usepackage{pstricks}
\usepackage{amsmath}
\usepackage{amsfonts}
\usepackage{amssymb}
\usepackage{eucal}
\usepackage[left=2cm,right=2cm,top=2cm,bottom=2cm]{geometry}
\pagestyle{empty}
\DeclareMathOperator{\tr}{Tr}
\newcommand*{\op}[1]{\check{\mathbf#1}}
\newcommand{\bra}[1]{\langle #1 |}
\newcommand{\ket}[1]{| #1 \rangle}
\newcommand{\braket}[2]{\langle #1 | #2 \rangle}
\newcommand{\mean}[1]{\langle #1 \rangle}
\newcommand{\opvec}[1]{\check{\vec #1}}
\renewcommand{\sp}[1]{{\begin{split}#1\end{split}}}
\usepackage{graphicx}
\usepackage{multicol}
\usepackage{float}
\usepackage{lipsum}

\usepackage{listings}
\usepackage{color}
\usepackage{algorithm,algorithmicx,algpseudocode}

\definecolor{codegreen}{rgb}{0,0.6,0}
\definecolor{codegray}{rgb}{0.5,0.5,0.5}
\definecolor{codepurple}{rgb}{0.58,0,0.82}
\definecolor{backcolour}{rgb}{0.95,0.95,0.92}

\lstdefinestyle{mystyle}{
	backgroundcolor=\color{backcolour},   
	commentstyle=\color{codegreen},
	keywordstyle=\color{magenta},
	numberstyle=\tiny\color{codegray},
	stringstyle=\color{codepurple},
	basicstyle=\footnotesize,
	breakatwhitespace=false,         
	breaklines=true,                 
	captionpos=b,                    
	keepspaces=true,                 
	numbers=left,                    
	numbersep=5pt,                  
	showspaces=false,                
	showstringspaces=false,
	showtabs=false,                  
	tabsize=2
}
\usepackage{hyperref}

\lstset{style=mystyle}

\begin{document}
\pagestyle{plain}
 
\begin{center}
\textbf{\large Greedy Algorithms}\\
\textbf{Name}: Ankit Kumar Singh
\hspace{8cm}
\textbf{GaTech ID}: asingh821
\end{center}
\hrule
%%%%%%%%%%%%%%%%%%%%%%%%%%%%%%%%%%%%%%%%%%%%%%%%%%%%%%%%%%%%%%%%%%%%%%%%

\section{General Notes}
Build solution piece by piece always adding the next piece that gives most obvious and immediate benefit. Sometimes the greedy algorithms are used to find the approximate solutions to the problems as well. Some of the common greedy problems are:
\begin{itemize}
	\item Minimum spanning tree (because of cut property i.e. pick the lightest edge in the cut)
	\item Dijkstra's algorithm ()
	\item Huffman encoding (pick the 2 least frequent ones, combine and repeat)
	\item Horn's formula
	\item Fractional Knapsack (pick the one with highest value density from the remaining ones)
	\item Acitivity selection, Job scheduling problems ()
	\item Approximate solution to Set cover, TSP. 
\end{itemize}

\subsection{Minimum spanning tree}

\textbf{Krushkal} - Repeatedly add the next lightest edge that doesn't produce a cycle.
\newline
\textbf{Cut property}: if X is set of edges part of an MST T of graph G = (V, E). Any subset of nodes S for which X doesn't cross between S and V - S. Lightest edge across the partition be e. Then X U {e} is still set of edges part of some other MST. 

\begin{algorithm}
	\floatname{algorithm}{MST}
	\algrenewcommand\algorithmicrequire{\textbf{Input: }}
	\algrenewcommand\algorithmicensure{\textbf{Output: }}
	\caption{Krushkal}
	\label{alg: Krushkal's Minimum spanning tree}
	\begin{algorithmic}[1]
		\Require a connected undirected graph G = (V, E) with edge weights $w_{e}$
		\Ensure a MST defined by edges X
		\For{all u $\epsilon$ V}
			\State makeset(u)
		\EndFor
		\State X = \{\}
		\State sort the edges E by weights
		\State for all edges (u,v) in increasing order of weights
		\If{ find(u) != find(v)}
			\State add edge (u, v) to X
			\State union(u, v)
		\EndIf
		\State \textbf{return} $state$
	\end{algorithmic}
\end{algorithm}

We use Disjoint set to find the representative of elements and check the cycle. 
\begin{algorithm}
	\floatname{algorithm}{Disjoint set}
	\algrenewcommand\algorithmicrequire{\textbf{Input: }}
	\caption{Makeset}
	\label{alg:Makeset}
	\begin{algorithmic}[1]
		\Require x
		\State pi(x) = x
		\State rank(x) = 0
	\end{algorithmic}
\end{algorithm}


\begin{algorithm}
	\floatname{algorithm}{Disjoint set}
	\algrenewcommand\algorithmicrequire{\textbf{Input: }}
	\caption{Find}
	\label{alg:Find}
	\begin{algorithmic}[1]
		\Require x
		\If{x != pi(x)}
		\State pi(x) = find(pi(x))
		\EndIf
	\end{algorithmic}
\end{algorithm}


\begin{algorithm}
	\floatname{algorithm}{Disjoint set}
	\algrenewcommand\algorithmicrequire{\textbf{Input: }}
	\caption{Union}
	\label{alg:Union}
	\begin{algorithmic}[1]
		\Require x, y
		\State rx = find(x), ry = find(y)
		\If{rx == ry}
		\State return
		\EndIf
		\If{rank(rx) > rank(ry)}
			\State pi(rx) = ry
		\Else
			\State pi(rx) = ry
			\If{rank(rx) == rank(ry)} rank(ry) = rank(ry) + 1
			\EndIf
		\EndIf
	\end{algorithmic}
\end{algorithm}


Prims algorithm is just like dijkstra's algorithm with only difference is that priority queue ordering. In prims its the value of the node is the weight of the lightest incoming edge from the set S, where as in Dijkstra's it is the length of the entire path from that starting point. 

\begin{algorithm}
	\floatname{algorithm}{MST}
	\caption{Prims}
	\label{alg:}
	\begin{algorithmic}[1]
		\State for all u in G.V: cost(u) = 0, prev(u) = null
		\State pick a initial node x, set cost(x) = 0
		\State H = makequeue(G.V) with cost values as keys
		\While{H is not empty}
			\State v = deletemin(H)
			\For{edge (v,z) in G[v].E}
				\If{w(v, z) < cost(z)}
				cost(z) = w(v, z), prev(z) = v
				\EndIf
			\EndFor
		\EndWhile
	\end{algorithmic}
\end{algorithm}


\subsection{Huffman's encoding}
For the data with high variation in the frequencies of the alphabets involved, most frequent letter can be encoded with lesser length values. Huffman encoding is one such variable length encoding scheme. Encoder metadata needs to be passed in the message for the reciever to be able to decode. 

\begin{algorithm}
	\floatname{algorithm}{Huffman}
	\algrenewcommand\algorithmicrequire{\textbf{Input: }}
	\algrenewcommand\algorithmicensure{\textbf{Output: }}
	\caption{}
	\begin{algorithmic}[1]
		\Require An array of frequency f[1...n]
		\Ensure Encoding tree with n leaves
		
		\State let H be a priority queue of integers ordered by f
		\State for i = 1 to n: insert(H, i)
		\For{k=n+1:2n-1}
			\State i = deletemin(H), j = deletemin(H)
			\State create a node numbered k with children i, j
			\State f[k] = f[i] + f[j]
			\State insert(H, k)
		\EndFor
	\end{algorithmic}
\end{algorithm}




\section{Common Problems}
Some fo the common Greedy problems from different sources and their variants. 

\subsection{Dijkstra's algorithm}
To find the single source shortest path, we can use dijkstra's algorithm (needs positive weight. atleast no negative weight cycle.) Other algorithms which solves the same problem is Bellman ford (which allows for negative weight cycles). 
\begin{algorithm}
	\floatname{algorithm}{Single source shortest part}
	\caption{Dijkstra}
	\label{alg:Dijkstra}
	\begin{algorithmic}[1]
		\State for all u in G.V: cost(u) = 0, prev(u) = null
		\State pick a initial node x, set cost(x) = 0
		\State H = makequeue(G.V) with cost values as keys
		\While{H is not empty}
			\State v = deletemin(H)
			\For{edge (v,z) in G[v].E}
				\If{w(v, z) < cost(z)}
				cost(z) = w(v, z), prev(z) = v
				\EndIf
			\EndFor
		\EndWhile
	\end{algorithmic}
\end{algorithm}

\subsection{Activity selection Problem}
\textbf{Problem statement: } You are given n activities with their start and finish times. Select the maximum number of activities that 
can be performed by a single person, assuming that a person can only work on a single activity at a time. \newline
\textbf{Approach: }
\begin{itemize}
	\item Sort the activities according to their finishing time 
	\item Select the first activity from the sorted array and print it 
	\item Do the following for the remaining activities in the sorted array. If the start time of this activity is greater than or equal to the finish time of the previously selected activity then select this activity and print it. 
\end{itemize}



\subsection{Job Sequencing Problem}
\textbf{Problem statement: } Given an array of jobs where every job has a deadline and associated profit if the job is finished before the deadline. It is also given that every job takes a single unit of time, so the minimum possible deadline for any job is 1. Maximize the total profit if only one job can be scheduled at a time. \newline
\textbf{Approach: }
\begin{itemize}
	\item Sort all jobs in decreasing order of profit.
	\item Iterate on jobs in decreasing order of profit.
	\item For each job, do the following : 
	\subitem Find a time slot i, such that slot is empty and i < deadline and i is greatest. Put the job in 
	this slot and mark this slot filled. 
	\subitem If no such i exists, then ignore the job. 
\end{itemize}


\section{Textbook solutions}

\end{document}